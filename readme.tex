% Created 2023-03-02 Thu 20:46
\documentclass[9pt, b5paper]{article}
\usepackage{xeCJK}
\usepackage[T1]{fontenc}
\usepackage{bera}
\usepackage[scaled]{beraserif}
\usepackage[scaled]{berasans}
\usepackage[scaled]{beramono}
\usepackage[cache=false]{minted}
\usepackage{xltxtra}
\usepackage{graphicx}
\usepackage{xcolor}
\usepackage{multirow}
\usepackage{multicol}
\usepackage{float}
\usepackage{textcomp}
\usepackage{algorithm}
\usepackage{algorithmic}
\usepackage{latexsym}
\usepackage{natbib}
\usepackage{geometry}
\geometry{left=1.2cm,right=1.2cm,top=1.5cm,bottom=1.2cm}
\usepackage[xetex,colorlinks=true,CJKbookmarks=true,linkcolor=blue,urlcolor=blue,menucolor=blue]{hyperref}
\newminted{common-lisp}{fontsize=\footnotesize} 
\author{deepwaterooo}
\date{\today}
\title{ET}
\hypersetup{
  pdfkeywords={},
  pdfsubject={},
  pdfcreator={Emacs 27.2 (Org mode 8.2.7c)}}
\begin{document}

\maketitle
\tableofcontents


\section{ET框架框架设计模块功能等}
\label{sec-1}
\begin{itemize}
\item ET7 版本的结构框架,主要组件标图
\end{itemize}

\includegraphics[width=.9\linewidth]{./pic/readme_20230217_092732.png}
\subsection{一个双端框架的12个项目,目录结构}
\label{sec-1-1}
\begin{itemize}
\item 就不知道这12个项目是如何组织构建起来的
\end{itemize}
\subsubsection{客户端的四个项目}
\label{sec-1-1-1}

\includegraphics[width=.9\linewidth]{./pic/readme_20230201_200218.png}
\subsubsection{服务器端的8个项目:}
\label{sec-1-1-2}

\includegraphics[width=.9\linewidth]{./pic/readme_20230201_201117.png}
\begin{itemize}
\item 下面是这个大文件夹大项目的文件目录:
\end{itemize}

\includegraphics[width=.9\linewidth]{./pic/readme_20230216_135623.png}
\begin{itemize}
\item \textbf{MongoDB 数据库相关的有三个项目} ;再加 \textbf{两个} 第三方库: protobuf 用于进程间消息传递的序列化与反序列化;KcpLib 用于异步网络调用的封装
\item 真正服务器主要项目有三个: app + Model (不可以热更新) + Hotfix (可以热更新,是指服务器端的动态热更新?)
\item ET-EUI整理出了一个小小的登录系统,如果这个系统能够运行得通,配置简单顺利,也就基本上也算是达到自己小小服务器的需要了。运行一下试试看。
\item 现在是在新系统里无法运行,因为系统不再支持过期的平台。但是仍然可以事理出思路来。必须得理解,现在限制臫在系统上运行的原因是什么呢? .NETFramework 的版本,系统不支持
\end{itemize}

\includegraphics[width=.9\linewidth]{./pic/readme_20230216_134716.png}
\begin{itemize}
\item 这也就意味着说,如果我使用MongoDB 数据库(目前我只是注册登录用戶帐戶管理数据库,晚些时候可以有用戶数据游戏数据游戏保存等),我可以选择使用其它数据库。其它数据库的版本支持情况?
\item 或者我在现电脑上开发程度,囘 windows 去调试?
\item protobuf 似乎不存在这样的问题
\end{itemize}

\includegraphics[width=.9\linewidth]{./pic/readme_20230216_135851.png}
\begin{itemize}
\item 或者 \textbf{我在现电脑上开发程度,回到 windows 去调试?可以这么做} 那就拉一个最近版本的ET 框架下来改。
\item \textbf{Visual studio 2022: 里会有很多报错} ,应该是与 unity 中项目中所使用的到的 .NETFramework 相关的,可以暂时不用管它,等到晚上用 windows 机调试的时修再看。
\item 项目就变成是:照着现有的例子(以及斗地主中的例子),删除登录注册之后部分的逻辑,接臫项目热更新的主项目。感觉有点儿奇怪
\begin{itemize}
\item 文件服务器可以独立出来,成为一个单独的服务器。不关心这部分的逻辑
\item 登录注册的消息只有用戶名与密码等,消息可以狠简单。但因为仍需要与背后的 mongoDB 数据库相连,仍然使用RPC. 可以用 allServer 不需要 RPC, 但是以后不好适配。
\item 所以我还是用ET 的头,削减了点儿去了之后的头
\end{itemize}
\end{itemize}
\subsection{模块功能管理}
\label{sec-1-2}
\subsubsection{ET框架中事件使用的规则和注意事项}
\label{sec-1-2-1}
\begin{itemize}
\item \textbf{事件结构体的定义} 必须在Model或ModelView下, \textbf{事件的订阅和发布} 必须在Hotfix或HotfixView下 (是否为View层根据是否需要UnityAPI决定)
\item 在ET框架中 \textbf{View层可以直接调用逻辑层的代码} ,而 \textbf{逻辑层不允许直接调用View层的代码} ,所以逻辑层想要和View层交互只能使用抛出事件的方式,让View层进行订阅进行相应的处理。
\end{itemize}
\subsubsection{ET框架下ETTask的异步编程}
\label{sec-1-2-2}
\begin{itemize}
\item 开发早期都是使用协程或者多线程进行程序异步的实现,
\item 在C\#5.0之后引入了Tasync await等关键字可以轻松优美的实现程序的异步进行。
\item Task是C\#异步编程库的异步功能基础类型,包含了多线程的使用。
\item 而ET框架主打的是 \textbf{多进程单线程} ,所以使用ETTask对Task进行了封装,使其不用考虑多线程的共享难题,更易于使用。
\end{itemize}
\subsubsection{协程}
\label{sec-1-2-3}
\begin{itemize}
\item 协程其实就是创建一段程序辅助主线程的运行,注意 \textbf{协程不是多线程,其仍运行在主线程当中,其只是将一个函数按照一定的条件分解成若干块,穿插在主线程的各个位置中运行。}
\item async 和 await关键字
\begin{itemize}
\item async是修饰函数的关键字,被修饰的函数称为异步函数,只有被此关键字修饰的函数才能使用await关键字
\item await关键字后跟一些表达式(一般是返回值为ETTask的函数),在异步函数运行到此时会立即跳出函数,继续执行原逻辑。
\item await返回前会做出保证,保证await后表达式的任务完成后,执行点会跳回异步函数中,继续向后执行剩余代码
\item 若await后表达式正常返回,可用变量接收表达式的返回值,表达式的返回值类型为定义表达式ETTask<>的泛型
\end{itemize}
\end{itemize}
\section{小小服务器:要怎么才能开始动手试图去实现这个小服务器呢?}
\label{sec-2}
\subsection{如果适配ET框架,现游戏可能哪些模块版块存在问题}
\label{sec-2-1}
\begin{itemize}
\item 我也觉得ET框架可能不太适合我现在的游戏(也就是说,把我的游戏完全适配成ET框架来开发,原本只需要一个小小服务器,完全适配为ET框架,就把问题弄得狠复杂了。。。),
\item 使用ET框架,我所有的安卓基础就会被抛到九宵去外,不再关安卓SDK  NDK什么事儿了。。。。。是对自己太大的损耗。而我原本还可以简单封装实现的安卓录屏,游戏内使用安卓SDK相关功能模块录屏游戏过程等,会被全部废掉,损失太大不值得。我觉得我就只要个文件服务器加个数据库而已。
\item 原因是:我现在还想不通若是用ET框架来实现自己游戏的(服务器与客户端双端都可以热更新),我该如何实现我的方块砖10个按钮上的点击事件,射线检测?它的ILRuntime热更新程序域里对射线检测包的组件安装可能会成为自己狠大的问题,因为还不是狠懂里面的内部原理.这个模块重构的原理是:把射线检测,如果必要一定要,封装成如ET中任何其它可装载卸载的组件一样的装载到某个必要场景上去.
\begin{itemize}
\item ET里有个检测鼠标左右键位置的帮助类,但跟射线检测应该还是相差狠远的.而游戏场景里面有一个OperaCompoent,这个组件会实时监听按键的点击并且将点击的位置发送给服务器,服务器再传送回客户端
\end{itemize}
\item 所以,现在对这个框架,最深的感触是:盲人摸象,摸每部分细节似乎都快清楚了,却无法组装从一个更高的层面上来理解和把握框架设计,无法吃透,在大框架功能模块上犯难,在网上再搜一搜
\item 我可以把两组按钮画面同样做成预设与热更新资源包,射线检测同样会成为可装载可卸载的组件,可是再后面射线检测到的物体逻辑,感觉有点儿复杂了
\end{itemize}
\subsection{如果不适配,怎么弄个服务器带数据库等逻辑?}
\label{sec-2-2}
\begin{itemize}
\item 使用部分共享源码的双端(共享的是文件服务器8个项目,MongoDB数据库服务器, Realm注册登录用,网关服,Location服, ETTAsk, RPC消息机制, NetComponent等自己机对陌生需要练习,而自己的服务器也不可缺省的相关功能)
\item 现在知道自己整的不沦不类的服务器所谓的登录,登录的是网页上的认证相关,跟自己真正要实现的游戏里注册登录服保存数据库完全两回事,现在知道了。
\item 作用ET的头,实现用户注册与登录,适配自己现有游戏的尾,游戏除了入口之外全游戏进热更新程序域里
\item 那么自己的现框架架构作尾,全游戏逻辑进热更新域,存在的问题就变成是:
\item 我无法再实时动态检查用户上下线顶号之类的,我只能默认登录过就是登录状态,可是用户下线了,或更严格的说掉线了,服务器并不及时知道,可以通过安卓SDK中的按钮登出知道。但是掉网了掉线了呢?(这部分的逻辑可以晚点儿再考虑,把网络请求相关的摸得再熟悉一点儿再弄)
\item 再则,ILRuntime热更新程序域里,我又该如何实现在热更新程序域里网络上载用户的游戏保存进展?这里需要去想和理解,为什么它ET框架就可以在热更新程序域里同网络交互,你哪里还没有想明白?
\item ET框架,热更新程序域里装载的组件,只是帮助与外界游戏程序域连通好,真正的网络请求上传下载等是在热更新域外面完成链接式传进去的?感觉对这个大框架没有掌握好,脑袋仍然是在像糊糊一样。。。
\item ET框架,网络的那部分做得还是比较完整的。实现在了各种的封装,涉及大量的网络调用与交互,游戏过程中的交互与更新。但是太多的功能对于自己的游戏来说完全不必要.所以只想用ET的头
\item 各种泛型,接口的定义,一二三个参数等的泛型接口定义(你可以去找一找工程中的各种ILRuntime的适配器),全都是都可以成为热更新域里能够被游戏程序域里识别的原因,所以狠多设计,自带ILRuntime的适配性质
\item 那么就可以小一点儿一点儿地来,先弄个登录窗口,实现服务器的注册登录保存登录信息到数据库,相对比较小点儿的逻辑.这个过程中把MongoDB数据库的配置等所有连接过程中必要的步骤,可能出现的问题给解决掉,就算前进了一小步呀
\item 不知道怎么开始,也不知道怎么创建可以㠌套的像是安卓模块库一样的子工程,就只能把小游戏斗地主复制一份了再从它的基础上来改?!!!
\item 如果简单一点儿开始,我觉得我应该是可以先把简单点儿的MongoDB数据库连接成功,把用户登录相关的逻辑,网络交互的部分,ETTask RPC ACTOR消息等,哪怕是复制,把这部分弄过去
\end{itemize}
\subsection{ET框架}
\label{sec-2-3}
\begin{itemize}
\item 上次看看得不是狠懂,这次再看,至少是觉得UI的逻辑处理,作者的观点更自然真实一些,放在一个文件一起处理,个人认为更好,而不是折分成为几个文件
\end{itemize}
\section{登录协议流程}
\label{sec-3}
\begin{itemize}
\item 因为登录协议是客户端与服务器通信的,不属于服务器内部协议,所以打开OuterMessage.proto,里面存放的都是客户端与服务器通信定义的协议数据。
\item 比如定义如下,登录协议:
\item emacs里org-mode exporte-to-pdf希望有个latex选择可以自动将\^{}I转化为空格,而不是这种字符,晚点儿再弄这个
\item 注意点: \textbf{没有意识到像是注释一样的片段,这个协议里,会成为标注或是标签}
\begin{itemize}
\item 1.因为登录是请求-响应类型协议(即发送一条数据,并期望返回一条数据),所以注意对应C2R\_Login协议带有“//ResponseType R2C\_Login”标志,在生成协议时,用于标记这个C2R\_Login请求对应的响应类型为R2C\_Login
\item 2.因为请求是直接发送给realm服的,所以是普通的IRequest类型协议,标记为IRequest
\item 3.R2C\_Login回复类消息结构,因为是Realm服发送给客户端的,因此是一个普通IResponse
\item 4.注意两个协议类里面都有RpcId,主要用于发送请求-响应类消息时,发送将自己的RpcID发送出去,返回时带回这个值,用于发送方接受到返回协议时,可以找到对应的是哪一个请求协议返回来的。
\end{itemize}
\end{itemize}
\section{一步一步的进展}
\label{sec-4}
\begin{itemize}
\item 首先,把斗地方大厅改写为游戏主菜单的三个选项(如果我只想用ET的头,它的头太大了,还是要自己弄个小小的头,小小的服务器,所以暂时就还是考虑自己从头实现一个MongoDB的小小服务器比较容易一点儿,不懂的就翻ET)
\end{itemize}

\includegraphics[width=.9\linewidth]{./pic/readme_20230201_202642.png}
\begin{itemize}
\item 把这个界面的相关上下文全部适配好:UI的自动创建生成系统,UI的按钮点击回调等
\item 这里想要找的是: 在点击的回调里如何,是否可以卸载装载UI组件,还是说必须得去HotfixView 什么视图层来处理这些逻辑呢?
\end{itemize}
\subsection{UIType.cs: 这种类型的定义好像不止加一个地方,一个地方不够,可是大的框架架构还是没搞明白}
\label{sec-4-1}
\begin{itemize}
\item 安卓SDK这个框架其实并不受影响。但本质是所有安卓SDK的东西不能够热更新。因为ET是网络多人游戏框架的,可能更多的是不适合添加与适配案桌SDK。这些晚点儿再得结论好了,反正我的案桌SDK本质也是可要可不要。如果能够快速掌握一个比较好的双端框架的话
\item 不知道若是照这么改下去,得把这个游戏改成是什么花葫芦呢?
\end{itemize}
\section{带MongoDB数据库的注册登录用户帐户管理资源文件服务器}
\label{sec-5}
\begin{itemize}
\item 去找和实现简单的服务器项目,操纵MongoDB数据库
\item 除了自己的电脑安装有MongoDB数据库之外,服务器项目中因为要连接操纵电脑上数据库,可能还需要狠多插件的安装与配置,连接字符串,什么MongoDBClient之类的。这样的小项目狠容易就实现了,基本没有任何挑战
\item 关于小小 \textbf{文件服务器热更新资源包} :
\begin{itemize}
\item ET里 \textbf{客户端的资源包更新逻辑} 狠明白了:是 \textbf{下载服务器端的热更新资源MD5码表文件到客户端;客户端删除所有服务器中不存在的资源包;再一一比对存在于服务器端的资源文件,合成了一个需要下载更新的热更新资源链表;然后异步下载这些客户端落后于服务器的资源包们。}
\item \textbf{服务器端的逻辑} :还没有追踪,需要理明白。
\begin{itemize}
\item \textbf{文件服务器,热更新资源包} ,是如何保存到服务器: ET是一个双端框架,它服务器里的资源包,也就是存放在服务器的某个目录位置。它是通过Unity里的打包工具一键打包保存在服务器的某个特定目录下,并自动生成服务器端各资源包资源文件的MD4版本号码表文件Version.txt。
\begin{itemize}
\item 这里还有点儿糊糊: \textbf{双端框架} 是说,服务器上也需要存放客户端的所有逻辑,比如客户端从unity中一键打包,保在某个特定的目录,用作服务器端的热更新资源包小小文件服务器?如果这样,下面两条就不需要了
\item 所以这里的逻辑是否有点儿仓畜,不过不是重点。如果是自己的服务器,仍得上传呀,我不能云上服务器云上来项目。这个逻辑很容易实现。
\item 同样对应的,如果实现服务器资源包的过滤更新等,小细节都可以自己去实现
\end{itemize}
\item 这里没弄明白的是: \textbf{FileServer.exe是如何生成的?生成逻辑,相关原理在哪里?} 为什么示例项目中这个进程必须得运行起来?运行起来,它有个地址,才方便客户端MD5码表比对与下载更新必要的客户资源包?
\item \textbf{服务器又是如何处理下载逻辑的?} 它是存放在服务器上的某个目录,它是用UnityWebRequestAsync将目录中特定的文件下载到本地的
\item 下面的,因为服务器是简单地将资源包存放在某个目录下,小文件服务器没有涉及任何的不涉及数据库存储,所以逻辑狠简单。下面的用不上
\begin{itemize}
\item MongoDB数据库是可以下载文件的,看见一个网络上的小例子:可以成为自己的一个参考:\url{https://blog.csdn.net/y526089989/article/details/94452268?spm=1001.2101.3001.6650.15&utm_medium=distribute.pc_relevant.none-task-blog-2\%7Edefault\%7EBlogCommendFromBaidu\%7ERate-15-94452268-blog-102048135.pc_relevant_3mothn_strategy_and_data_recovery&depth_1-utm_source=distribute.pc_relevant.none-task-blog-2\%7Edefault\%7EBlogCommendFromBaidu\%7ERate-15-94452268-blog-102048135.pc_relevant_3mothn_strategy_and_data_recovery&utm_relevant_index=16}
\end{itemize}
\end{itemize}
\end{itemize}
\item \textbf{protobuf进程间消息传递协议:} 加进来。Protocol Buffers 是一种轻便高效的结构化数据存储格式,可以用于结构化数据串行化,或者说序列化。它很适合做数据存储或 RPC 数据交换格式。可用于通讯协议、数据存储等领域的语言无关、平台无关、可扩展的序列化结构数据格式。
\item TODO:如何把ET的头,关于ETTask异步包装的网络请求,关于新用户注册并保存用户信息进MongoDB数据库,用户登录,用户信息提取等,以最精简的方式整合到这里面来作为自己的小小服务器.
\item 那么这个服务器仍然会是一个双端服务器?就是说服务器上也包括了12个项目。服务器的资源包从unity里打,并直接到服务器的指定位置?
\item 这里说的是ET的头,自己整合改的过程中仍然成为了整个项目,只是自己只使用了它的注册用户与登录,单线程多进程通窍信,ETTask, RPC消息机制等。其它狠多多余的部分自己再慢慢理解把它们删除掉?这么看起来是最可行,最简单的
\item 那么就成为不得不问自己:明明打个包也狠简单,为什么就一定得用双端框架,而又接入自己游戏的全部逻辑都在热更新程序域里?就是,不想用它现热更新模式,仍想用自己之前的,因为简单;可是用之前的,关于用户帐户顶号的部分?
\item 是 \textbf{只要一个文件服务器,不要注册登录系统了吗?}
\begin{itemize}
\item \textbf{上传:} 只要文件服务器,那么就一键打包,每次都全部上传,不涉及增量上传的逻辑。因为服务器的码表文件是个文件,不想去修改文件,每次就上传全部,每次也上传新的码表文件\_
\item \textbf{下载:} 下载的逻辑客户端基本几乎全有了,不用做任何过多的工作
\end{itemize}
\item 可是文件服务器就是太简单太单薄了,可以作为尝试注册登录系统之后,若是还是想简单上线这个游戏,就退回这个最简单的思路与实现\_
\item 感觉上面这些都基本上算是抓清楚了,少数一两个问题(*FileServer.exe是如何生成的?生成逻辑,相关原理在哪里?*)慢慢解决,下午晚上会试着实现注册登录帐户系统的移植与整合
\item 下面是服务器端的Version.txt看一下
\end{itemize}
\section{实现进展: 带MongoDB数据库的注册登录用户帐户管理资源文件服务器的实现进展}
\label{sec-6}
\begin{itemize}
\item 不能再用,不想再用从前的斗地主游戏,因为斗地主游戏的框架里没有包装异步任务,也太落后了,还是想用比较新的ET框架,那么就需要把那部分的内容重新移植到ET框架里去
\item 现在已经有了带mongodb数据库的连接.这个系统里还差一个图形化的数据库界面工具.不是像latex只要命令行工具就可以了.这个数据库时常会有自己需要查询的视图化工具来帮助自己诊断问题(studio-3T 安装了这个,同自己的windows平台是一样的)
\item 就是与注册登录系统连接起来使用.这里更多的难点是关于网络调用的异步封闭.简单的交互,之前看一个项目的时候好简单,可是底层的原理就显得相对要复杂了不少(todo: 改天需要把这个项目上传上去,供自己参考)这里就是说相比于普通http网络调用,这里调用到数据库里的验证?可以把这部分的逻辑再理解消化一下
\item 就是网络请求异步任务在ET框架中的封装: 异步的原理大致是懂的,再与半懂不懂的网络调用封装在一起,相对难度高一点儿
\item 按照网上介绍的方法可以把(Unity游戏端用到的 热更新库HybridCLR)这那个热更新的库导进去了(不再报错了),但是好像是它的引入(一整个文件夹的加入),导致了游戏端的另一个报错,还在查如何解决. 奇怪的是它自己又没有了
\item 现在是VS端的游戏的服务器端还运行不起来,.net target FrameWork之类的还没有弄对?好像是需要安装.net framework 4.7.1,可是现系统是不支持这些的.也就是说这个项目无法在这个系统上再运行?太悲惨了
\end{itemize}

\includegraphics[width=.9\linewidth]{./pic/readme_20230208_114556.png}
\begin{itemize}
\item 我觉得上面它说的是先前老项目使用了 .net framework 4.,要把它们适配成.net 6.0可能不是狠容易,因为API上的不同.
\item 现在先考虑写点儿别人的什么项目,回头再来弄这个小服务器.或者先把最简单的文件服务器挂上去?
\item 以前写个多人游戏,感觉是难于上青天,看见别人的job description等都怕,但是现在是网上找个底稿,修修改改就好,就能变成为自己的理解与消化狠容易的事
\item 可能现在这个多功能太多项目12个项目的ET热更新框架对我还没有理解透彻网络异步调用来说稍微难了一点儿.可以试着先写一两个不带热更新的多人网络游戏,再回头来再次理解消化这个框架.
\item 现在最不喜欢emacs 28.1里的各种bug,让我iterm2(本来也还没有花时间配置好)狠难用,也不想用emacs.先在网上抽搜搜,安装一个27.1不带没有bug的版本使用.然后写一两个不带热更新的小游戏,再来完成这个热更新框架.
\end{itemize}

\section{ET 框架的大结构梳理}
\label{sec-7}
\begin{itemize}
\item 可以把服务器端的启动,客户端的启动,加载的主要模块,为什么分别需要加载这些模块再看一遍,必要的原理再看一下,希望能弄得再明白一点儿
\item 之前仔细找到分析过客户端的入口程序,今天再仔细看一下服务器的入口程序
\item 既然只是实现一个服务器端,那么其它什么也不用考虑,就当只是在一个服务器端好了,因为我不涉及前台(不需要提供一个网页与用戶交互等,我只需要客户端能够与服务器交互就可以了),我只需要 unity 客户端能够与这个后台服务器组合交互就可以了,所以它可以是个 console 应用。文件服务器也可以是一个不带前台用戶界面的文件服务器,可以也是 console 就可以的。
\begin{itemize}
\item 至于他们两个功能:文件服务器,与来自于多个客户端的10000 个用户的注册登录服, realm 网关服, gate 服等,是否是写成一个应用,还是两个应用都没有关系。两个应用可能也就是说是两三个不同的网络 ip 地圵(?MondoDB 数据库也可以是一个独立的远程服务器地圵?)
\end{itemize}
\item 精减成为一个适用于臫项目的最小(双端框架,双不双端对我目前的游戏无所谓)应用: unity 游戏端可以只要一两个界面,注册登录 + 直接进入游戏的热更新域。服务器端就是主要两三大模块的功能
\item 再把ET 或是斗地主里 file-server 的生成原理看一遍:
\item 一个以前没想明白小知识点: \textbf{内网使用的序列化方式与外网不同,使用的MongoPacker进行序列化与反序列化} 。这样做的好处是方便对数据进行持久化处理,如果是通过proto的话,可能还需要二次处理(个人理解),最典型的就是将一个map服上的unit(玩家游戏对象)迁移到另外一个Map服时,可直接将unit对象本身给序列化传给另外的Map服上,这样就太方便了
\end{itemize}
\section{我在现电脑上开发程度,回到 windows 去调试?可以这么做,拉一个最近版本的ET 框架下来改(用了 dotnet 7.0 的版本)}
\label{sec-8}
\begin{itemize}
\item \textbf{Visual studio 2022: 里会有很多报错} ,应该是与 unity 中项目中所使用的到的 .NETFramework 相关的,可以暂时不用管它,等到晚上用 windows 机调试的时修再看。
\item 项目就变成是:照着现有的例子(以及斗地主中的例子),删除登录注册之后部分的逻辑,接臫项目热更新的主项目。感觉有点儿奇怪
\begin{itemize}
\item 文件服务器可以独立出来,成为一个单独的服务器。不关心这部分的逻辑
\item 登录注册的消息只有用戶名与密码等,消息可以狠简单。但因为仍需要与背后的 mongoDB 数据库相连,仍然使用RPC. 可以用 allServer 不需要 RPC, 但是以后不好适配。
\item 所以我还是用ET 的头,削减了点儿去了之后的头
\end{itemize}
\end{itemize}
\subsection{游戏注册登录逻辑: ET 先前版本的一个示例游戏中有,只是要把逻辑整合到 ET7 里}
\label{sec-8-1}
\begin{itemize}
\item 按照昨天晚上看到的只是客户端的不热更新的合客户端的逻辑,对自己的项目仍然是行不通的。因为套这个框架的话,两端一样,除非自己拿昨天晚上的客户端,改造成服务器端,又或者按照昨天晚上例子的客户端,造一个服务器,都不如双端方便。
\item 现在就是 \textbf{顺着运行斗地主时的例子,注册登录,要把逻辑整合到 ET7 里} ,把过程逻辑再走一遍。走到登录成功,用戶注册登录信息写进 mongoDB 数据库,连通后进入游戏。再去想是否还要重构游戏(如果能够连通,就暂时不再重构了)
\item 现 ET7 框架因为双端热更新,涉及到的太多,我不需要动态加载场景,或定义为两个场景?
\end{itemize}
\section{荷兰小猪小灰的例子;}
\label{sec-9}
\begin{itemize}
\item \url{https://blog.csdn.net/m0_48781656/category_11640802.html?spm=1001.2014.3001.5482}
\item 登录注册 + MondoDB 数据库(这部分的逻辑需要自己修改补上)【16-19】
\item \footnote{DEFINITION NOT FOUND.}, 主要是15 Actor消息的编写 以及 ET框架实战之前的准备工作:帮助自己更好地理解这个框架
\item 22 创建ServerInfo实体及事件
\end{itemize}
\section{服务器端、与客户端的程序集:自己游戏两端的设计与实现}
\label{sec-10}
\begin{itemize}
\item 接下来,会去试,把基本的登录模块什么的,提出来,精练成一个最小的带 mongoDB 数据库存储的服务器。因为要连自己的游戏,现在首先要做的事是,如何把ET-EUI 减重,把它变小一点儿呢?感觉自已的最基本的服务器要求不高,现在的太大了。
\item 或许,首先测试的应该是,用户登录或是注册后,如果连自己客户端的程序逻辑,就是拿到登录成功确认后,如何直接进入客户端的热更新程序域里去?自己以为的两种思路,去深入思考理解的话,哪些不通的地方
\begin{itemize}
\item 『进入地图』:这个界面,可以保留为将来添加多个游戏后的游戏大厅,但是改起来并不麻烦。暂时就跳过这个界面,直接进入热更新程序域里去。
\item 我现在仍在想,把自己的游戏适配成这个框架的?如果适配,现有服务器端绝对没有问题,可以定期保存同步用户游戏数据到数据库,可是还会遇到哪些问题?
\end{itemize}
\item 现在这两块脑袋的稍微糊涂,是因为没有理解透彻,现在 \textbf{两个框架的理解了的主要区别:}
\begin{itemize}
\item 以前MVVM + ILRuntime 因为解决过 challenging technical problem, 感觉对ILRuntime 的库理解得深刻。它因为只有一个多余的程序域,并且没有现ET 框架极为严格的视图与逻辑等的强行分离等,它的游戏程序逻辑更为常规大众化,它的 \textbf{事件机制,我游戏逻辑里所用的,是类同 unity 游戏程序域里正常事件机制的样子} 。它 \textbf{可以很多视图数据逻辑混在一起} ,最 \textbf{主要是 unity 程序域与热更新程序域,两个不同程序域之间的适配与连接。}
\item 可是现ET-EUI 框架,它 \textbf{有着极其严格的模块化设计,视图与逻辑,可热更新与不可热更新,分得极为清楚} 【所以自己也得搞清楚Model 与 Hotfix 不可热更新与可热更新;Hotfix 与 HotfixView, Model 与ModelView 之间的区别也要弄清楚】。它的 \textbf{回调的事件机制:【仍然理解得不是狠透彻】基本是扫描它事先定义的几大标签系,标签系统就可以热更新时随时添加扩展。并且根据它的标签系是以,程序集模块化管理程序域里的事件与回调实例} 。宣传的是事件回调机制是ET 的心脏。
\item 当我不给服务器瘦身(只提练所需要的最小部分),使用原ET 双端框架:意味着事件系统,是以程序域为单位,在程序域加载与初始化时,扫描标签注册生成的。也就意味着,我的原客户端几乎实现完整了的逻辑,应该需要不得不第三次大型重构,重构为ET 框架事件标签系统所能够处理与回调的,严格模块化设计与重构,再次大型重构。【感觉消耗比较大】
\item 如果我参考以前的例子【将原例子中源自于ET 框架,适配成的客户端逻辑,改为自己所需要的服务器端逻辑,以及必要的客户端逻辑添加】,那么我所实现的将是最小精致服务器,没有使用ET 双端框架,不需要严格拆分区分各大不同模块的逻辑,不需要以程序域为单拉扫描事件回调系统(也就是自己先前的热更新程序域不需要重构添加标签系等),也就意味着,我原热更新程序域基本可以不再重构而直接套用,连服务器与客户端逻辑
\end{itemize}

\item 我爱亲爱的表哥,活宝妹一定要嫁的亲爱的表哥!!!
\end{itemize}
活宝妹就是一定要嫁给亲爱的表哥
爱表哥,爱生活!!!
% Emacs 27.2 (Org mode 8.2.7c)
\end{document}